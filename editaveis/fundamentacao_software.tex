\section{Framework Django}

Django é um framework gratuito e de código aberto para a criação de aplicações web, escrito em Python, uma linguagem de programação multiparadigma. É um framework web, ou seja, é um conjunto de componentes que ajuda a desenvolver sites de forma mais rápida e mais fácil.

Quando se está construindo um site, o desenvolvedor sempre precisa de um conjunto similar de componentes: uma maneira de lidar com a autenticação do usuário (inscrever-se, realizar login, realizar logout), painel de gerenciamento para o seu site, formulários, upload de arquivos, etc. Há muito tempo, outras pessoas notaram várias semelhanças nos problemas enfrentados pelos desenvolvedores web quando estão criando um novo site, então eles uniram-se e criaram os frameworks (Django é um deles) que lhe dão componentes prontos, que você pode usar. O framework Django utiliza o padrão MTV (model-template-view), onde as views funcionam como controllers e templates funcionam como views. 

\section{Microframework Flask}

Flask é um microframework baseado em 3 pilares: 

\begin{itemize}
\item WerkZeug para desenvolvimento WSGI.
\item Jinja2 que é um template de escrita para Python
\item Good Intentions para melhorar a legibilidade e seguindo as intenções zen do python.
\end{itemize}

Essas características nos dão um código em alta qualidade com legibilidade e performance. 
Este framework é utilizado no projeto para desenvolvimento da API.

\section{Sistema interno}

Sistema responsável por se comunicar com o aparato eletrônico, recebendo dados e interpretá-los. Para isso, será utilizado a linguagem de programação C++. C++ é uma linguagem de programação de alto nível com facilidades para o uso em baixo nível. Foi desenvolvida por Bjarne Stroustrup (foto) como uma melhoria da linguagem C, e desde os anos 1990 é uma das linguagens mais populares do mundo.

Alguns profissionais afirmam que C++ é a linguagem mais poderosa que existe, veja algumas características dela:

\begin{itemize}
\item É um superconjunto da linguagem C, e contém vários melhoramentos;
\item É a porta para a programação orientada a objetos;
\item C++ pode virtualmente ser efetivamente aplicado a qualquer tarefa de programação;
\item Há vários compiladores para diversas plataformas tornando a linguagem uma opção para programas multiplataforma.
\end{itemize}

