O banco de baterias fornece cerca de 16 V e 5000 mAh. Como o Quirby irá trabalhar em corrente contínua, a energia disponível, e portanto, sua autonomia serão baseados pela quantidade de corrente disponibilizada pela fonte. Sendo assim, equação abaixa demonstra a autonomia teórica do robô.

\begin{equation}
E = Ah = 5000 mAh = 5 Ah
\end{equation}

Sendo E a energia disponível ao sistema. 
A corrente necessária para manter o pleno funcionamento do sistema equivale a cerca de 3,26 A. Sendo assim, é possível calcular a autonomia teórica em horas do robô.

\begin{equation}
    Autonomia = \frac{5 Ah}{3,26 A}
\end{equation}
\begin{equation}
    Autonomia = 1,53 h = 92 min
\end{equation}

Portanto, desconsiderando as perdas por efeito Joule e por outros fenômenos adversos, a autonomia teórica do robô seria de aproximadamente 92min.

Agora ao considerar o motor de sucção do projeto como uma máquina de fluxo, é possível utilizar o conceito do balanço de energia para determinar o fluxo mássico do ar.

\begin{equation}
Q-W = m\left(\Delta e+\frac{\Delta P}{\rho}\right)
\end{equation}

Onde:
\begin{itemize}
	\item{Q:} é a quantidade de calor por unidade de tempo que entra na máquina;
	\item{W:} é a quantidade de trabalho que a máquina recebe ou fornece para o fluido, neste caso o ar;
	\item{m:} é a quantidade total de massa por unidade de tempo que adentra e sai da máquina. Como não há acumulo de fluido, logo “m” será constante;
	\item{$\Delta$e:} é a variação de todas as energias internas do fluido.
	\item {$\Delta$P e \(\rho\):} representam respectivamente, a variação de pressão causada pelo rotor e a densidade do fluido.

Dessa forma, podemos simplificar a equação adotando algumas condições, sendo elas:

	\item Não há entrada ou saída de calor da máquina; 
	\item O trabalho será fornecido pela bateria, logo terá valor negativo;
	\item Não há variação significativa da energia interna do fluido.

\end{itemize}

Assim, a equação se resume da seguinte forma:

\begin{equation}
W = m\frac{\Delta P}{\rho}
\end{equation}

Portanto, aplicando os valores de W, $\Delta$P e \(\rho\) calcula-se o m:


\begin{equation}
m=W\ast\frac{\rho}{\Delta P}
\end{equation}

Considerando:

\begin{itemize}
	\item W $=$ 24W;
	\item \(\rho\) $=$ 1,2041 kg/m\textsuperscript{3}
	\item {$\Delta$P} $=$ 1600 Pa.
\end{itemize}

Temos:

\begin{equation}
m=24\ast\frac{1,2041}{1600}
\end{equation}

\begin{equation}
m = 0,018\frac{kg}{s} = 18\frac{g}{s}    
\end{equation}

\vspace{1\baselineskip}
Portanto, o Quirby possui um fluxo teórico de aspiração de partículas em cerca de 18 g/s.